\documentclass[10pt, letterpaper, acro-macros]{python-problem}

\usepackage{url}

%% Acronyms
\DeclareAcronym{daq}{short = daq, long = Data Acquisition Systems}

\newcommand{\oski}{\emph{OskiBot 5000}\xspace}

\problemset{2}
\date{}

\begin{document}

\maketitle

\section*{Rules}

\begin{itemize}
  \item Use Python 3 for this exercise (it is not possible within Python 2).

  \item
    Certain ways of writing operations are equivalent, and hence count as only
    one implementation method. For example, list comprehensions are equivalent
    to for loops, and temporary value creation is the same as chaining.

    \begin{pycode*}{gobble=6}
      ## The following two are equivalent implementations
      a1 = [x**2 for x in range(5)] ## This is the python 3 range
      a2 = []
      for x in range(5):
          a2.append(x**2)

      ## The following two are equivalent implementations
      temp = 'Hello World!'.lower()
      b1 = temp.split()
      b2 = 'Hello World!'.lower().split()
    \end{pycode*}

  \item Globals are not allowed.

  \item Reviewing ideas and definitions on the internet is encouraged,
        however, please attempt in all earnestness to develop the source
        code on your own. Using solutions or even similar ideas on the
        internet can rob you of the learning process

  \item Write up the solutions as if it were a real homework assignment.

\end{itemize}


%%%%%%%%%%%%%%%%%%%%%%%%%%%%%%%%%%%%%%%%%%%%%%%%%%%%%%%%%%%%%%%%%%%%%%%
%                              Problems                               %
%%%%%%%%%%%%%%%%%%%%%%%%%%%%%%%%%%%%%%%%%%%%%%%%%%%%%%%%%%%%%%%%%%%%%%%

\section*{Problems}

There once was a girl named Clare, who was the most talented engineer of
them all. She specialised in \daq cards, which she could use to slay even
the most difficult digital acquisition dragons.
\marginpar{Ryan is a bit tired}

One day, she decided she wanted to create a battle droid for the
\textsc{first} robotics competition, complete with treads and a cool flame
decal (for +2 damage points). She started out by writing her normal \daq
routine for creating a waveform, which went along the following lines.

\begin{pycode*}{gobble=2}
  def waveform(frequency, phase):
      ''' Return discrete samples of a waveform, at an offset of phase.  '''
      # Code code code

      ## return a new phase, which points to the place we stopped taking
      ## waveform samples at.
      return waveform_points, phase

  phase = 0
  while True:
      waveform_points, phase = waveform(frequency, phase)
      sendToDaq(waveform_points)
\end{pycode*}

However, she was tired of this method, as it required her to carry around a
variable all through her code. Having just learned Python, she read about an
incredible capability she didn't have access to in C: first class functions!
She pondered to herself:

\begin{enumerate}
  \item What is a first class function? Write down three unique examples of
        functions being used as first class objects.

  \item A similar concept named higher order functions came up in her search.
        What do those do? How do they relate to first class functions?

  \item She had met Bill from problem set 1. Was \code{talkToBill} a first
        class function or a higher order function?

  \item Two similar concepts are nested functions and closures. Describe
        these two programming concepts, and what is different between them.

  \item (Bonus) What is the performance cost (qualitatively) for using
        first class functions in Python? When should a developer worry about
        these problems in practice?
\end{enumerate}

``Hmm, this seems pretty neat!'' she thought. She decided to try a few crazy
things.

\begin{enumerate}[resume]
  \item Create a function named \code{gThenf}, which takes in two
        functions $f$, $g$ as input (where $f$ and $g$ both only take in one
        argument) and returns a new function. This new function takes in one
        input $x$ and returns the result of first applying $g$ to $x$, and
        then applying $f$ to the result from $g$.

  \item This function is widely known by another name. What is it?


  \item Why are the arguments $f$, $g$ in that order?

  \item Write a new function(s) that does the same as \code{gThenf}, but
        performs this on any number of inputs.
\end{enumerate}

These examples were cute, but she was hungry to use the computer for
something that she couldn't do easily on paper.

\begin{enumerate}[resume]
  \item Make a new function \code{walkAndApply}, that takes in a function
        $f$ and applies it to every item in an iterable $x$. This should
        work for any type that is iterable. For example, $x$ could be a
        list, a tuple, a set, a dictionary, a priority heap, a tree, etc.
        The returned value should be a list.

  \item Use this function to apply a variety of functions (of your choosing)
        to a single value.

  \item What is the common name of this function?
\end{enumerate}

Given what she has learned, Clare decides to turn her attention back to
\oski, her lovable bear robot of destruction. Part of the
\emph{first} robotics competition has the judges try to break the robot by
sending it arbitrary waveforms that are outside of the operational voltage
range of the motors.

\begin{enumerate}[resume]
  \item The waveforms given by the judges come as functions. Devise a way to
        monitor these waveform functions to print ``\oski is shutting
        down to protect itself'' when the waveform exceeds a certain value
        defined at runtime (values cannot be hard coded). The result should
        have the same calling structure as the original waveforms given by
        the judges.

  \item Devise a waveform function that returns a random value any time it is
        called. The random value must be allowed to exceed some runtime
        defined error threshold, as above.

  \item Using decorator notation apply your protection function to this
        new waveform function in the prior question.

\end{enumerate}

With her robot now safe from harm, she looks back at her original
implementation of \code{waveform} and figures out a way to simplify her
system so she does not have to pass around phase all the time. In this new
system, each analog output channel on her \daq card analog is given
a function \code{f(number\_samples)}, where \code{number\_samples} is the
number of samples that the \daq wants to take from the waveform
function \code{f}.

\begin{enumerate}[resume]
  \item Devise a waveform generator function that fits in Clare's new
        \daq architecture, where the phase does not have to be
        passed around explicitly. Additionally, this method should be able
        to make waveforms of different parameters. For example, a sine
        waveform generator should be able to keep track of the phase and
        work for any frequency. Additionally the same call should be able to
        attach to an arbitrary number of channels, each of which may require
        a different parameter set (for example, 4 channels running sine
        waves, each with some random initial phase offset and frequency).
        Choose two types of waveforms to generate (sine wave, ramp, triangle
        wave, etc).

  \item Decorate these functions will two different types of protection
        checks. One should be for when the waveform exceeds some bounded
        region (for example, $\pm 3.3\,\mathrm{V}$), while the other should
        check for when the rate of change between waveform samples exceeds
        some value (the slew rate is higher than the system can handle).
\end{enumerate}

With these modifications complete, \oski is ready to obliterate
\emph{ArborTreeSissyBot}.

For those in the Conolly lab, answer the following extra questions.

\begin{enumerate}[resume]
  \item How do the fourth set of questions (those about protection) relate
        to the real time system?
  \item How to the fifth set of questions (those about analog output)
        relate to the real time system?
\end{enumerate}


%%%%%%%%%%%%%%%%%%%%%%%%%%%%%%%%%%%%%%%%%%%%%%%%%%%%%%%%%%%%%%%%%%%%%%%
%                              Solutions                              %
%%%%%%%%%%%%%%%%%%%%%%%%%%%%%%%%%%%%%%%%%%%%%%%%%%%%%%%%%%%%%%%%%%%%%%%
\beginsolution

\newpage
\section*{Solutions}

\begin{enumerate}
  \item I am the first solution bitches!
\end{enumerate}

\end{document}
